\begin{frame}{D 树上的数 by \itshape xiaolilsq}
		如果每条边上最后的数字都大于 $0$,那么有一种众所周知的构造方式使得每条边上的数字都为断掉这条边后剩下两个连通块中大小较小的那个连通块大小(具体而言就是找到重心,然后各个子树中互相配对操作),这样操作下来在 $n\ge 3$ 的情况下出现次数最多的数字为 $1$,其出现次数为叶子数量. \pause
		
		事实上在 $n\ge 3$ 的时候显然每个叶子连上去的边数字只能为 $1$,所以我们的构造已经达到最优了. \pause
		
		额外需要注意的是 $n=2$ 的时候最优答案为 $1$,比叶子数量少 $1$.
		
	\end{frame}

\begin{frame}{D 树上的数 by \itshape xiaolilsq}
	接下来考虑有 $0$ 的情况,那么所有操作都不能跨过钦定为数字 $0$ 的边,所以我们完全可以把这些数字为 $0$ 的边断掉,然后剩下若干个连通块,数字 $1$ 的出现次数为这些连通块叶子数量之和减去大小为 $2$ 的连通块. \pause
	
	我们可以证明,如果剩下多个连通块,我们把它们连接起来数字 $1$ 的出现次数不会更多,而数字 $0$ 的出现次数会减少,所以存在最优解满足数字非 $0$ 的那些边组成的是一个连通块. \pause
	
	不妨钦定数字 $1$ 的出现次数,也就是钦定叶子数量,我们的目的是尽可能减少数字 $0$ 的出现次数,也就是说我们要找到一个满足叶子数量限制的尽可能大的连通块,这使用长链剖分贪心做就行了. \pause
	
	同样需要注意 $n\le 3$ 可能要特殊处理.
\end{frame}