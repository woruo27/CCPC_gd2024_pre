\documentclass[dvipsnames]{ctexbeamer}

\usepackage{bm}
\usepackage{mathtools}
\usepackage{stmaryrd}
\usepackage{color, xcolor}
\usepackage{tikz}
\usetikzlibrary{arrows, positioning}

\usepackage{fourier}
\usepackage{nicefrac}
\usepackage{ulem}

\usepackage[thicklines]{cancel}%cancel
\usepackage{array,multicol,multirow}
\usepackage{diagbox}
\usepackage{geometry}

\usetikzlibrary{arrows, positioning, quotes}




\usepackage[
    type={CC},
    modifier={by},
    version={4.0},
    % imagemodifier={-80x15},
    lang={chinese-utf8},
]{doclicense}

% \usepackage[orientation=landscape,size=custom,width=16,height=9,scale=0.5,debug]{beamerposter}

\newtheorem{remark}[theorem]{注记}

\DeclareMathOperator{\mex}{mex}
\newcommand{\me}{\mathrm{e}}

\newcommand{\Dname}{三染色}
\newcommand{\sfT}{\mathsf T}

% \usetheme{Berkeley}
\usetheme{metropolis}
% \usecolortheme{spruce}
\usefonttheme{professionalfonts,structurebold}

\title{2024年广东省大学生程序设计竞赛(GDCPC) \\ 暨CCPC广州邀请赛}
% \subtitle{Selected Topics in Graph Enumeration {\color{red} (draft)}}
\author{}
\institute{清华大学学生算法协会}
\date{2024 年 5 月 26 日}

\begin{document}

\frame{\titlepage}

\frame{
    感谢 ~\textsc{Mys.C.K.}, \textsc{liuzhangfeiabc}, \textsc{Itst}, \textsc{SpiritualKhorosho}, \textsc{EliminateSpace}, \textsc{xiaolilsq}, \textsc{gyh-20}, \textsc{JohnVictor36}, \textsc{Renshey}, \textsc{QAQAutoMaton} 负责命题和验题工作.

}

% \section[Outline]{}
\frame[allowframebreaks]{
    \tableofcontents[sections={1-5}]
    \framebreak
    \tableofcontents[sections={6-10}]
}


\section{A 田字格 by \itshape Itst}
\begin{frame}{A 祥子的工作 by \itshape {johhhh}}
\begin{block}{题目描述}
已知$X_1,X_2,\cdots,X_n$满足一个离散的概率分布,且独立同分布,求$\text{Pr}[X_1+X_2+\cdots+X_n \ge a]$
\end{block}
\begin{block}{算法}
签到题。

设$p_{i,j}$表示前$i$天恰好赚$j$分的概率,有递推公式$p_{i,j}=\frac{1}{101} \sum_{k=0}^{100} p_{i-1,j-k}$,拿个前缀和优化就可以.

时间复杂度$O\left(100 n a \right)$
\end{block}

丰川祥子的姓名第一个字拼起来(丰祥)意外地像一个中国名字

\end{frame}

% \section{B 腊肠披萨 by \itshape SpiritualKhorosho}
% \begin{frame}{简要题意}
	
	定义 $LCPS\left(s, t\right)$ 为最长的 $r$ 使得 $s_1 \cdots s_{|r|} = t_{|t| - |r| + 1} \cdots t_{|t|} = r$. 求

	$$\sum_{i=1}^L \sum_{j=1}^L C^{\left|LCPS\left(s_i, s_j\right)\right|} \bmod{P}.$$

	$1\le L, \sum_{i=1}^L \left|s_i\right|\le 3\times 10^6, 2\le C < P < 2^{30}$.

\end{frame}

\begin{frame}{观察}
	
	虽然 $LCPS$ 看上去没有什么规律, 但是注意到输入串的所有前缀和所有后缀的数量都是与总串长相等的 (如果去重则更少) , 可以考虑用后缀去查是否存在对应的前缀, 并更新相应 $\left(s_i, s_j\right)$ 的答案. 对前缀建立 Hash 表, 即可 $\Theta(1)$ 查询每个后缀. \pause

	这个做法有一个问题: 一个前缀可能对应多个原串, 如果需要统计每对 $\left(s_i, s_j\right)$ 的 $LCPS$ 长度再计算答案, 复杂度可能会爆炸. \pause

	因此, 我们需要设计一种方法, 通过查询到的前缀直接计算该前缀的贡献, 并且想办法容斥保证每对 $\left(s_i, s_j\right)$ 只有匹配上的最长前缀/后缀的原始贡献计入答案. 

\end{frame}

\begin{frame}{容斥}
	
	考虑什么时候一个串 $s_i$ 会有多个前缀被另一个串 $s_j$ 的相应后缀匹配上. 假设已知 $s_{i,1}\cdots s_{i,k} = s_{j,\left|s_j\right|-k+1}\cdots s_{j, \left|s_j\right|}$ 且 $s_{i,1}\cdots s_{i,l} = s_{j,\left|s_j\right|-l+1}\cdots s_{j, \left|s_j\right|}$, 
	其中 $l<k$, 那么不难发现:\pause

	$s_{i,1}\cdots s_{i,l}$ 应该是 $s_{i,1}\cdots s_{i,k}$ 的一个 border!\pause

	这启发我们可以对每个串跑 KMP, 用 fail 来处理容斥. 注意到 border 的 border 仍是 border, 这个容斥的形式非常好看. \pause

	一个前缀会在每个 border 处统计一次该 border 的贡献. 因此, 我们只需要将当前前缀的原始贡献减去其最长 border 的原始贡献. 把一个前缀的原始贡献拆分成其 border 贡献之和后, 我们就可以在查询后缀时直接把对应贡献乘上前缀出现次数计入答案即可. \pause
	
	总复杂度 $O\left(\sum_{i=1}^L \left|s_i\right|\right)$. 注意 $C$ 的幂次可以递推, 如果每次暴力算会多个快速幂的 $\log$, 有可能无法通过本题. 

\end{frame}

\section{C DFS 序 by \itshape EliminateSpace}
\frame
{
  \frametitle{C 小班课 by \itshape gyh20}

 	考虑我们如何处理一次询问. \pause

	任意时刻,一些小班是满员的,其余的未满员. 我们可以假设每个人都去到自己意向最高的小班,直到某一个小班变满,可以发现,总共只会有 $m$ 次小班变满的过程. 每一次我们可以直接二分查找找到这个位置. \pause

	具体二分查找时,我们处理的询问为,在区间 $l\sim r$ 中有多少人满足其在小班集合 $S$ ($S$ 为当前未满员的小班构成的集合)中优先度最高的是 $x$. 可以发现这个区间信息量是 $O(2^m m)$ 的,使用线段树维护,总复杂度为 $O(2^m m n\log n+qm^2\log n)$.

}


% \section{D 拨打电话 by \itshape SpiritualKhorosho}
% \begin{frame}{D 树上的数 by \itshape xiaolilsq}
		如果每条边上最后的数字都大于 $0$,那么有一种众所周知的构造方式使得每条边上的数字都为断掉这条边后剩下两个连通块中大小较小的那个连通块大小(具体而言就是找到重心,然后各个子树中互相配对操作),这样操作下来在 $n\ge 3$ 的情况下出现次数最多的数字为 $1$,其出现次数为叶子数量. \pause
		
		事实上在 $n\ge 3$ 的时候显然每个叶子连上去的边数字只能为 $1$,所以我们的构造已经达到最优了. \pause
		
		额外需要注意的是 $n=2$ 的时候最优答案为 $1$,比叶子数量少 $1$.
		
	\end{frame}

\begin{frame}{D 树上的数 by \itshape xiaolilsq}
	接下来考虑有 $0$ 的情况,那么所有操作都不能跨过钦定为数字 $0$ 的边,所以我们完全可以把这些数字为 $0$ 的边断掉,然后剩下若干个连通块,数字 $1$ 的出现次数为这些连通块叶子数量之和减去大小为 $2$ 的连通块. \pause
	
	我们可以证明,如果剩下多个连通块,我们把它们连接起来数字 $1$ 的出现次数不会更多,而数字 $0$ 的出现次数会减少,所以存在最优解满足数字非 $0$ 的那些边组成的是一个连通块. \pause
	
	不妨钦定数字 $1$ 的出现次数,也就是钦定叶子数量,我们的目的是尽可能减少数字 $0$ 的出现次数,也就是说我们要找到一个满足叶子数量限制的尽可能大的连通块,这使用长链剖分贪心做就行了. \pause
	
	同样需要注意 $n\le 3$ 可能要特殊处理.
\end{frame}

\section{E 循环赛 by \itshape ckw20}
\frame
{
  \frametitle{E 计数题 by \itshape rsy}

 	定义无向图 $G$ 的权值 $f(G)$ 为 $G$ 中所有连通块大小的平方和.

	对于 $1 \sim n$ 的所有错排 $p$,求 $f(G(p))$ 之和对 $998244353$ 取模后的结果.

	 $2 \le n < 998244353$.

}
\frame
{
  \frametitle{E 计数题 by \itshape rsy}

 	考虑计算每个连通块(轮换)的贡献: $f(n)=\sum_{k=2}^n k^2 \binom n k (k-1)! g(n-k)$,其中 $g(n)$ 为长度为 $n$ 的错排个数. \pause

	$$f(n)=\sum_{k=2}^n k^2 \binom n k (k-1)! g(n-k)$$
	$$=\sum_{k=2}^n k^2 \frac {n!}{(n-k)!k!} (k-1)! g(n-k)=n!\sum_{k=2}^n k \frac {g(n-k)}{(n-k)!}$$

	记 $A(x)=\sum_{k=2}^\infty kx^k =\frac {x}{(1-x)^2}-x$,$G(x)=e^{\ln \frac 1 {1-x}-x}=\frac {e^{-x}}{1-x}$,则 $f(n) =n![x ^ n]A(x)G(x) = n![x ^ n]\left(\frac {xe^{-x}}{(1-x)^3} - \frac {e^{-x}}{1-x}+e^{-x}\right)$. \pause

	可以拆成三部分计算,利用求导不难得到每部分的递推式. 使用分块打表或整式递推加速计算即可.

}

\section{F 图 by \itshape xiaolilsq}
\begin{frame}{题目大意}
	考虑 $1$ 以及大于 $\sqrt{n}$ 的质数都必须要单独放在一个集合内,而我们通过构造的方式证明其它数字最多额外剩下一个数,其余数都可以放在大小大于 $1$ 的集合中.
\end{frame}

\begin{frame}{题解}

	
\end{frame}

\section{G Menji 和 gcd by \itshape gyh-20}
\begin{frame}{题目大意}
	求 $\max\limits_{L\leq x<y\leq R}\gcd(x,y)$。
\end{frame}

\begin{frame}{题解}

分答案 $\geq \sqrt r$ 和答案 $\leq \sqrt r$ 的情况。\pause

答案可以为 $x$ 当且仅当 $\left\lfloor \dfrac{r}{x}\right\rfloor-\left\lfloor \dfrac{l-1}{x}\right\rfloor\geq 2$,可以 $O(1)$ 检查。\pause

于是 $\leq \sqrt r$ 的情况可以枚举。\pause

否则可以枚举 $\left\lfloor \dfrac{r}{x}\right\rfloor$ 求出最小的合法的 $x$。\pause

时间复杂度 $O(\sqrt r)$。
	
\end{frame}

\section{H 小班课 by \itshape gyh-20}
\include{problems/H}

\section{I 不等式 by \itshape JohnVictor36}
\begin{frame}{题目大意}
	给定 $n,m$,以及 $m$ 个形如 $a_{x_i}\ge a_{y_i}+a_{z_i}(1 \le i \le m)$ 的条件。问是否有一组正整数 $(a_1,a_2,\cdots,a_n)$ 满足所有条件,并且 $a_1+a_2+\cdots+a_n \le 10^{9}$。如果有,输出 $a_1+a_2+\cdots+a_n$ 的最小值;如果无解,输出 $-1$。
\end{frame}

\begin{frame}{题解}

考虑连边 $x_i\to y_i,x_i \to z_i$. 注意到如果这个图里面有环,那么环上的最小值一定不满足对应的要求;如果图里面无环,找出拓扑序,按照拓扑序依次贪心确定每个数的最小可能值即可. 如果超过了 $10^9$ 直接退出就不会溢出. 时间复杂度 $O(n+m)$. 
	
\end{frame}

\section{J 另一个计数问题 by \itshape Renshey}
\begin{frame}{题目大意}
	给定一个 $n - 1$ 个点的无向图,点的编号为 $2 \sim n$。对于所有的 $2 \le u < v \le n$,边 $(u, v)$ 存在当且仅当 $v$ 是 $u$ 的正整数倍。定义 $f(u, v)$ 表示 $u$ 与 $v$ 是否连通:当 $u, v$ 连通时 $f(u, v) = 1$,否则 $f(u, v) = 0$。求:

	$\left(\sum_{u = 2} ^ {n - 1} \sum_{v = u + 1} ^ n f(u, v) \cdot u \cdot v\right) \bmod {998244353}$

\end{frame}

\begin{frame}{题解}
注意到,对于所有 $x \le \lfloor \frac{n + 1}{2} \rfloor$ 的结点 $x$,$x$ 与 $2x$ 间有一条边,而 $2x$ 与 $2$ 之间有一条边,因此所有 $2 \sim \lfloor \frac{n + 1}{2} \rfloor$ 范围内的结点连通。而对于 $x > \lfloor \frac{n + 1}{2} \rfloor$ 的结点 $x$,$x$ 不与 $2 \sim \lfloor \frac{n + 1}{2} \rfloor$ 中的结点连通当且仅当 $x$ 为素数。因此整个图仅包含若干个单独的素数结点与一个大的连通块。\pause

考虑计算不连通的点对之间的贡献。不难发现,只需要求出所有大于 $ \lfloor \frac{n + 1}{2} \rfloor$ 的素数的和与平方和。使用 Min\_25 筛或分块打表均可以。

\end{frame}


\begin{frame}{}
    \begin{center}
        \Large 感谢倾听!
    \end{center}
\end{frame}

\end{document}
